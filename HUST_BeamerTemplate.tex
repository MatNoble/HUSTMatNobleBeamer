%# -*- coding:utf-8 -*-
\documentclass[10pt,aspectratio=43,serif]{beamer}		
%设置为 Beamer 文档类型,设置字体为 10pt,长宽比为16:9,字体为 serif 风格

%%%%-----导入宏包-----%%%%
\usepackage{matnoble}			%导入 matnoble 模板宏包
\usepackage{microtype}
\usepackage[BoldFont,SlantFont,CJKchecksingle]{xeCJK}
\usepackage{amsthm,amsmath,amsfonts,amssymb,bm}   %导入数学公式所需宏包
\usepackage{color}			 %字体颜色支持
\usepackage{graphicx,hyperref,url}
%\hypersetup{colorlinks,linkcolor=red} 
\usepackage{metalogo}	% 非必须
\usepackage{xunicode}%-----------------------------------------------------------提供Unicode字符宏
\usepackage{xltxtra}%------------------------------------------------------------提供了针对XeTeX的改进并且加入了XeTeX的LOGO
\usepackage{tikz}
\usetikzlibrary{positioning,backgrounds}
\usetikzlibrary{fadings}
\usetikzlibrary{patterns}
\usetikzlibrary{calc}
\usetikzlibrary{shadings}
\usepackage{tabularx,multirow,multicol,longtable}
\usepackage{tabu}
\setbeamertemplate{caption}[numbered]
\setbeamerfont{caption}{size=\small}
%% 上文引用的包可按实际情况自行增删
%%%%%%%%%%%%%%%%%%	

%================================== 设置中文字体 ================================%
\setCJKmainfont{Adobe Heiti Std}%------------------------------------------------设置正文为黑体
\setCJKmonofont{Adobe Song Std}%-------------------------------------------------设置等距字体
\setCJKsansfont{Adobe Kaiti Std}%------------------------------------------------设置无衬线字体
\setCJKfamilyfont{zxzt}{FZShouJinShu-S10S}
\setCJKfamilyfont{FZDH}{FZDaHei-B02S}
%================================== 设置中文字体 ================================%

%================================== 设置英文字体 ================================%
%\setmainfont[Mapping=tex-text]{TeX Gyre Pagella}%--------------------------------英文衬线字体
\setmainfont[Mapping=tex-text]{Helvetica}%--------------------------------英文衬线字体
\setsansfont[Mapping=tex-text]{Trebuchet MS}%------------------------------------英文无衬线字体
\setmonofont[Mapping=tex-text]{Courier New}%-------------------------------------英文等宽字体
\newfontfamily\Arial{Arial}
%================================== 设置英文字体 ================================%

%================================== 设置数学字体 ================================%
\usepackage[T1]{fontenc}
\usepackage{concmath}
%================================== 设置数学字体 ================================%

\beamertemplateballitem		%设置 Beamer 主题

%%%%------------------------%%%%%
\catcode`\。=\active         %或者=13
\newcommand{。}{.}				
%将正文中的“。”号转换为“.”。中文标点国家规范建议科技文献中的句号用圆点替代
%%%%%%%%%%%%%%%%%%%%%

%%%%----首页信息设置----%%%%
\title[ 公众号: 数学家园 ]{华中科技大学 Beamer 模板}
\subtitle{——这里是副标题}			
%%%%----标题设置

\author[张\quad 振]{
  %\vskip5em
  张\quad 振 \\
  \medskip
  %{\small \href{mailto:z-matrix@qq.com}{z-matrix@qq.com}} \\
  {\small \href{https://github.com/MatNoble/}{github.com/MatNoble}}
}

%%%%----个人信息设置
  
\institute[HUST]{
  %偏微分方程数值解 \\ 
  %\medskip
  {\normalsize 数学与统计学院, 华中科技大学}}
%%%%----机构信息

\date[Jul. 13 2018]{
	%\today
  2018年7月13日
}
%%%%----日期信息
  
\begin{document}

\begin{frame}
\titlepage
\end{frame}				%生成标题页

\section{提纲}
\begin{frame}
\frametitle{提纲}
\tableofcontents
\end{frame}				%生成提纲页

\section{介绍}
\begin{frame}
  \frametitle{介绍}

  \begin{itemize}
    \item {编译方式}
    \vskip0.5em
	    \begin{itemize}
	    	\item  推荐安装完整版的 TeXLive
	    	\item 使用 \XeLaTeX 编译
	    \end{itemize}
    \vskip0.5em
    \item 请参考 \LaTeX 和 Beamer 用户文档 
    \vskip1em
    \item 行内数学公式示例 $\sin^2 \theta + \cos^2 \theta = 1$
    \vskip1em
    \item {行间数学公式示例 
    \begin{equation}
	    y_{1}=\int \sin x\, {\rm d}x
    \end{equation}	 } 
    \item 下载地址: \href{https://github.com/MatNoble/}{github.com/MatNoble}
  \end{itemize}
\end{frame}

\section{内置环境}
\begin{frame}
  \frametitle{内置环境}
	\begin{block}{Slides with \LaTeX}
	    Beamer offers a lot of functions to create nice slides using \LaTeX.
	  \end{block}
	
	  \begin{block}{The basis}
	    内部使用以下主题
	    \begin{itemize}
	      \item split
	      \item whale
	      \item rounded
	      \item orchid
	    \end{itemize}
	  \end{block}
\end{frame}

\begin{frame}
  \frametitle{带数字列表}
	 \begin{enumerate}
	    \item This just shows the effect of the style
	    \item It is not a Beamer tutorial
	    \item Read the Beamer manual for more help
	    \item Contact me only concerning the style file
	  \end{enumerate}
\end{frame}

\begin{frame}
\frametitle{图片}
	\begin{figure}
		\includegraphics[height=6cm]{figures/logo.jpg}
		\caption{欢迎扫码关注}
    \end{figure}
\end{frame}

\section{结论}
\begin{frame}
  \frametitle{结论}

  \begin{itemize}
    \item Easy to use
    \item Good results
  \end{itemize}
\end{frame}

\section{参考文献}
\begin{frame}{参考文献}
\begin{thebibliography}{99} 
\bibitem{zhao1} Yi~Zhao, {\sl An introduction to X}, Sep.~15, 2015
\bibitem{qian2} Er~Qian, San~Sun, 
Phys.\ Lett.\ A {\bf xx}, 2xx (20xx)   
\bibitem{li4} Si~Li, Phys.\ Rev.\ C {\bf xx}, 5xx (20xx) 

\end{thebibliography}
\end{frame}

\section{提问}
\begin{frame}{致谢}
\begin{center}
	\begin{tikzpicture}
	\node[above,xscale=1.2,yscale=1.4]{\Huge\bfseries 欢迎各位老师批评指正!};
	\node[xscale=1.2,above,yscale=-1.4,scope fading=south,opacity=0.2]{\Huge\bfseries 欢迎各位老师批评指正!};
	\end{tikzpicture}
\end{center}
\end{frame}

\end{document}