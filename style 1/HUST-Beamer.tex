% # -*- coding:utf-8 -*-
% xelatex

\documentclass[10pt,aspectratio=43,serif]{beamer}		
% 设置为 Beamer 文档类型,设置字体为 10pt,长宽比为4:3(16:9),字体为 serif 风格
\usepackage[hustc]{cnlogo}
\usepackage{matnoble-beamer-1}			%导入 matnoble 模板宏包

%%%%----首页信息设置----%%%%
\title[ XX答辩 ]{华中科技大学 Beamer 模板}
\subtitle{——这里是副标题}			
%%%%----标题设置

\author[MatNoble(HUST)]{\href{https://matnoble.me/about/}{MatNoble}}
\institute[HUST]{\normalsize 数学与统计学院, 华中科技大学}
\date[July 21, 2020]{\today}

\logo{\hustclogo[0.1]}
  
\begin{document}

\begin{frame}
    \titlepage
\end{frame} %生成标题页

\section{提纲}
\begin{frame}
    \frametitle{提纲}
    \tableofcontents
\end{frame} %生成提纲页

\section{介绍}
\begin{frame}
    \frametitle{介绍}

    \begin{itemize}
        \item {编译方式} \vskip0.5em
        \begin{itemize}
            \item 推荐安装完整版的 TeXLive
            \item 使用 \XeLaTeX 编译
        \end{itemize}
        \vskip0.5em
        \item 请参考 \LaTeX 和 Beamer 用户文档 \vskip1em
        \item 行内数学公式示例 $\sin^2 \theta + \cos^2 \theta = 1$
        \vskip1em
        \item {行间数学公式示例
          \begin{equation}
              y_{1}=\int \sin x\, \dif x
          \end{equation}	 } 
        \item 下载地址:
        \href{https://github.com/MatNoble/HUSTMatNobleBeamer}{github.com/MatNoble}
    \end{itemize}
\end{frame}

\section{内置环境}
\begin{frame}
    \frametitle{内置环境}
    \begin{block}{Slides with \LaTeX}
        Beamer offers a lot of functions to create nice slides using
        \LaTeX.
    \end{block}
	
	  \begin{block}{The basis}
              内部使用以下主题
              \begin{itemize}
                  \item split
                  \item whale
                  \item rounded
                  \item orchid
              \end{itemize}
	  \end{block}
      \end{frame}

\begin{frame}
    \frametitle{带数字列表}
    \begin{enumerate}
        \item This just shows the effect of the style
        \item It is not a Beamer tutorial
        \item Read the Beamer manual for more help
        \item Contact me only concerning the style file
    \end{enumerate}
\end{frame}

\begin{frame}
    \frametitle{图片}
    \begin{figure}
        \includegraphics[height=6cm]{figures/logo.jpg}
        \caption{欢迎扫码关注}
    \end{figure}
\end{frame}

\begin{frame}
    \frametitle{图片}
    \begin{figure}[H]
        \centering
        \begin{subfigure}{.48\textwidth}
            \centering
            % include first image
            \includegraphics[width=.6\linewidth]{logo}
            \caption{\em 子图 1}
            \label{fig:v21}
        \end{subfigure}
        \begin{subfigure}{.48\textwidth}
            \centering
            % include second image
            \includegraphics[width=.6\linewidth]{google}
            \caption{\em 子图 2}
            \label{fig:v22}
        \end{subfigure}
        \caption{\em 并列两张图}
        \label{fig:v2}
    \end{figure}
\end{frame}

\begin{frame}
    \frametitle{表格}
    \begin{table}[ht]
        \centering
        \caption{\em 这是表格}
        \vskip 0.1in
        \label{table}
        \begin{tabular}{c|cccc}
          \hline
          \hline
          \rule{0pt}{3ex}
          序号 & 姓名 & 年龄 & 学号 & 性别 
                                      \rule[-1.2ex]{0pt}{0pt} \\\hline
          001 & *  &  *  & *  & * \\ 
          002 & *  &  *  & *  & * \\
          003 & *  &  *  & *  & * \\      
          004 & *  &  *  & *  & * \\
          \hline
          \hline 
        \end{tabular}
    \end{table}
\end{frame}

\section{结论}
\begin{frame}
    \frametitle{结论}

    \begin{itemize}
        \item Easy to use
        \item Good results
    \end{itemize}
\end{frame}

\section{参考文献}
\begin{frame}{参考文献}
    \begin{thebibliography}{99}
        \bibitem{zhao1} Yi~Zhao, {\sl An introduction to X}, Sep.~15,
        2015
        \bibitem{qian2} Er~Qian, San~Sun, Phys.\ Lett.\ A {\bf xx},
        2xx (20xx)
        \bibitem{li4} Si~Li, Phys.\ Rev.\ C {\bf xx}, 5xx (20xx)
    \end{thebibliography}
\end{frame}

\section{提问}
\begin{frame}{致谢}
    \begin{center}
	\begin{tikzpicture}
            \node[above,xscale=1.2,yscale=1.4]{\Huge\bfseries 欢迎各位
              老师批评指正!}; \node[xscale=1.2,above,yscale=-1.4,scope
            fading=south,opacity=0.2]{\Huge\bfseries 欢迎各位老师批评指
              正!};
	\end{tikzpicture}
    \end{center}
\end{frame}

\end{document}
